\documentclass[12pt]{article}
 \usepackage{amsmath}
 \usepackage{mathtools}
 \usepackage{accents}

 \newcommand*{\dt}[1]{%
  \accentset{\mbox{\large\bfseries .}}{#1}}
  
 \begin{document}
 
 \title{Problem Set 1}
\author{Alexa Villaume\\ 
Radiative Transfer} 
 
\maketitle

\noindent \textbf{Problem 1 (RL 1.2)} Photoionization is a process in which a photon is absorbed by an atom (or molecule) and an electron is ejected. Any energy at least equal to the ionization potential is required. Let this energy be $h\nu_0$ and let $\sigma_\nu$ be the cross section for photoionization. Show that the number of photoionizations per unit volume and per unit time is,

\begin{equation*}
4\pi n_a \int_{\nu_0}^{\infty} \frac{\sigma_\nu J_\nu}{h\nu}d\nu = cn_a\int_{\nu_0}^\infty\frac{\sigma_\nu u_\nu}{h\nu}d\nu,
\end{equation*}

\noindent where $n_a$ is the number density of atoms.\\

\noindent \textbf{Answer} \\

\noindent \textbf{Problem 2 (RL 1.7)} 

\begin{itemize}

\item Show that if stimulated emission is neglected, leaving only two Einstein coefficients, an appropriate relation between the coefficients will be consistent with thermal equilibrium between the atom and a radiation field of a Wien spectrum, but not of a Planck spectrum. 

\textbf{Answer} \\


\item Re-derive the relation between the Einstein coefficients by imagining the atom to be in thermal equilibrium with a neutrino field (spin 1/2) rather than a photon field (spin 1). \textbf{Hint:} Neutrinos are Fermi-Dirac particles and obey the exclusion principle. In addition, their equilibrium intensity is given by,

\begin{equation*}
I_\nu = \frac{2h\nu^3 / c^2}{{\rm exp}\left[h\nu / kT\right] + 1}.
\end{equation*}

\textbf{Answer} \\


\end{itemize}

\end{document}